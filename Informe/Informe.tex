\documentclass{article}
\usepackage{graphicx}
\usepackage[margin=2cm]{geometry}
\usepackage[hidelinks]{hyperref}
\usepackage{color}

\begin{document}

\begin{titlepage}
  \begin{center}
    \huge
    \itshape\bfseries
    Proyecto Moogle!\\
    \medskip
    William Daniel Barrios Bosque, C-113\\
    \medskip
   18 de julio de 2023\\
  \end{center}
\end{titlepage}

\tableofcontents

\section{Para ejecutar mi programa}
  \begin{enumerate}
    \item Crear en la carpeta del proyecto una carpeta llamada "Content" en la que debe poner los documentos
      \begin{itemize}
        \renewcommand{\labelitemi}{$\diamond$}
        \item Cada documento debe ser de extensión ".txt" y su nombre debe ser algo como\\
        palabras\_en\_minusculas\_separadas\_por\_guiones\_bajos.txt
      \end{itemize} 
    \item Abrir un terminal en la carpeta del proyecto
    \item Ejecutar el comando "make dev" en linux o si es windows pase directamente al paso siguiente
      \begin{itemize}
        \renewcommand{\labelitemi}{$\diamond$}
        \item De no tener la herramienta make ejecutar el comando "dotnet run --project MoogleServer" para linux o "dotnet watch run --project MoogleServer" para windows
      \end{itemize}
    \item Mientras presiona "Ctrl" de clic donde dice "http://localhost:5285", esto abrirá una página web en su navegador donde puede buscar, si es windows se abrira directamente en el navegador
  \end{enumerate}
\begin{center}
\end{center}
 
 \begin{frame}
 \item Como aun no termino el proyecto no tengo un informe completo de la situacion, Readme.Lyx son los avances hasta el momento
 \end{frame}

\end{document}
